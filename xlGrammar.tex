
\documentclass[conference]{IEEEtran}
%\usepackage{cite}

% correct bad hyphenation here
\hyphenation{}

\begin{document}

\title{The Grammar of Spreadsheets}

\maketitle


\begin{abstract}

The abstract goes here.
\end{abstract}

\IEEEpeerreviewmaketitle


\section{Introduction}

spreadsheets as programming:
- cell calculation chains as lines of code - size, example
- avg lines of code per spreadsheet in the enron dataset

motivation: why we need a defined grammar
- parsing spreadsheet formulas, analysis, extracting metrics, finding smells, exploring the structure of spreadsheets
- refer to papers needing/referring to a simple grammar
- benefit of a simple grammar and documentation for spreadsheet users
- to explore/uncover smelly syntactical constructs that are within the grammar coverage

challenges:
- official grammar for Excel not defined. Only through datasets analysis can we discover all the possible ways eg to 
express a reference, partly because excel is very forgiving
- grammar changes with different Excel versions, eg regular expressions support, labels in formulas now expanded to cell
references, updates in the built-in functions list

our approach:
development of grammar starting from previous work (reference to simple grammar) and analyzing the two datasets, enriching the
grammar to support everything

\section{Spreadsheet Formula Grammar}

\subsection{Lexical analysis}
terminals/lexical tokens, table with 
Token|Informal description of pattern|Regular expression(?)|Example lexemes

\subsection{Syntax analysis}
non-terminal production rules
EBNF
syntax/railroad diagrams
% see \usepackage[rounded]{syntax}

\subsection{Precedence and ambiguity}

\section{Evaluation}
datasets, scantool

\section{Wacky Grammar}

References
- [1]!Hub\_Consolidation
- Functions returning references (INDEX|OFFSET|INDIRECT|???) SUM(A1:INDEX(A:A,1,1:1))
- External UDFs RIGHT([1]!SheetName(),(LEN([1]!SheetName())-20))
- Prefixed right limits =SUM(Deals!F9:'Deals'!F16)
- This thing we do not know what it is =BLP|M!'INDU Index,[PX\_close\_5d]'

Array formulas
- IF(OR(MONTH(GQ2) = {3,9}), IF(ABS(SUM(GQ223) - GQ225)>0.001,1,0), 0)

Arrays used as arguments
- LARGE((F38,C38),1)

Reserved names
- Add-in functions C9*\_xll.HEAT(B9,C9)
- \_xlnm.* =SUMIF('150000'!A:A,\_xlnm.Print\_Titles,'150000'!C:C)

Smelly grammar constructs + how often they are used
- Implicit intersect operator =COUNT(A1:A10˽A1:A5)
- Complex ranges =SUM(I8:namedRange:M8)
- Combinations of the above SUM((Total\_Cost Jan):(Total\_Cost Apr.))

\section{Discussion and Limitations}
Grammar for Excel version ???
Not covering array formulas (?)

\section{Comparison against the LibreOffice grammar}

\section{Related work}

\section{Conclusion}
The conclusion goes here.




\section*{Acknowledgment}
The authors would like to thank...






%\bibliographystyle{IEEEtran}
% argument is your BibTeX string definitions and bibliography database(s)
%\bibliography{IEEEabrv,../bib/paper}
%
% <OR> manually copy in the resultant .bbl file
% set second argument of \begin to the number of references
% (used to reserve space for the reference number labels box)
\begin{thebibliography}{1}

\bibitem{IEEEhowto:kopka}
H.~Kopka and P.~W. Daly, \emph{A Guide to \LaTeX}, 3rd~ed.\hskip 1em plus
  0.5em minus 0.4em\relax Harlow, England: Addison-Wesley, 1999.

\end{thebibliography}


\end{document}


