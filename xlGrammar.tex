
\documentclass[conference]{IEEEtran}

%\usepackage{cite}

% correct bad hyphenation here
\hyphenation{}

\usepackage{eurosym}
\usepackage{amsfonts}
\usepackage{cite}

%\usepackage{appendix}

% Tables
\usepackage[table,xcdraw]{xcolor}
\usepackage{booktabs}
\usepackage{pbox}
\renewcommand{\arraystretch}{1.2} 
%\renewcommand*\cmidrule{} % No table middle lines
%\renewcommand{\arraystretch}{1.5} % Additional spacing with no middle lines
%\renewcommand*\cmidrule{\hdashline[1pt/2pt]}% Dashed middle lines
\renewcommand*\cmidrule{\midrule[0.001em]} % Thin table middle lines
%\renewcommand*\cmidrule{\midrule} % Thick table middle lines

%Images
\usepackage[pdftex]{graphicx}
\graphicspath{ {}{img/} }
\DeclareGraphicsExtensions{.pdf,.jpg,.png}

% Syntax package configuration
\usepackage[rounded]{syntax}
\setlength{\grammarparsep}{4pt plus 1pt minus 1pt}

\newcommand{\todo}[1]{\textbf{TODO: #1}}
\newcommand{\ignore}[1]{}

\begin{document}

\title{A LALR(1) Grammar for Excel Formulas}

\maketitle


\begin{abstract}

The abstract goes here.
\end{abstract}

\IEEEpeerreviewmaketitle


\section{Introduction}

spreadsheets as programming:
- cell calculation chains as lines of code - size, example
- avg lines of code per spreadsheet in the enron dataset

motivation: why we need a defined grammar
- parsing spreadsheet formulas, analysis, extracting metrics, finding smells, exploring the structure of spreadsheets
- refer to papers needing/referring to a simple grammar
- benefit of a simple grammar and documentation for spreadsheet users
- to explore/uncover smelly syntactical constructs that are within the grammar coverage

challenges:
- official grammar for Excel defined, but too complex/implementation specific for formula analysis
- excel is very forgiving
- grammar changes with different Excel versions, eg regular expressions support, labels in formulas now expanded to cell
references, updates in the built-in functions list

our approach:
development of grammar starting from previous work (reference to simple grammar) and analyzing the two datasets, enriching the
grammar to support everything

\section{Spreadsheet Concepts}
explain intersection, precedence?, ...

\section{Design Decisions}
trade off between ambiguity, simplicity of the grammar and usefulness of the parse trees

\section{Spreadsheet Formula Grammar}

\todo{Add appendix with excel function names}

All strings and characters are case-insensitive, unless otherwise noted.

\subsection{Lexical analysis}

Table \ref{table:tokens} contains an overview of the tokens produced by the scanner in our parser. The rules are given in a very simple regular expression language, but could be easily adapted for a scanner which does not support regular expression tokens. 

Our tokens also require the scanner to support token priorities. Removing the necessity for token priorities is possible by altering the tokens and production rules, but makes the grammar more complicated and the resulting tree harder to use. Leaving the tokens as-is but not providing priorities will still result in a usable grammar, but expect errors on edge cases.

\begin{table*}
\label{table:tokens}
\caption{Lexical tokens used in our grammar}
\begin{tabular}{@{}lllll@{}}
\toprule
Token Name & Description & Regular Expression & Priority & Example \\
\midrule
BINOP & Binary Operator & + $\mid$ - $\mid$ / $\mid$ * $\mid$ \textasciicircum $\mid$ \textless $\mid$ \textgreater $\mid$ = $\mid$ \textless= $\mid$ \textgreater= $\mid$ \textless \textgreater & 0 & + \\
BOOL & Boolean literal & TRUE $\mid$ FALSE & 0 & TRUE \\
CELL & Cell reference & \$? [A-Z]+ \$? [0-9]+ & 2 & $A$1 \\
ERROR & Error literal & \#NULL! $\mid$ \#DIV/0! $\mid$ \#VALUE! $\mid$   \#REF! $\mid$ \#NAME? $\mid$ \#NUM! $\mid$ \#N/A & 0 & \#REF! \\
FILE & External file reference & \textbackslash[ [0-9]+ \textbackslash] & 5 & [1]\\
add normal file names \\
FUNCTION & Excel built-in function & (Any entry from Appendix X) \textbackslash( & 5        & SUM( \\
HORIZONTAL\_RANGE & Range of rows & \$? [0-9]+ : \$? [0-9]+ & 0 & 1:4 \\
MULTIPLE\_SHEETS & Multiple sheet references &
[A-Z0-9]+ : ([A-Z0-9\_.]+ $\mid$ ' ([A-Z0-9\_ !@\#\$\%\textasciicircum{}\&*()\-+=\{\}:;$\mid$\textless\textgreater,./?\textbackslash{}\textbackslash{}] $\mid$ '')+  ') !
& 1 & Sheet1:'Escape '' 5'! \\
NAMED\_RANGE & Named range & [A-Z\_][A-Z0-9\_.]* & -2 & MYNAME \\
NAMED\_RANGE\_PREFIXED & \begin{tabular}[c]{@{}l@{}} Named range which starts with \\ a string  that could be another token \end{tabular} & (TRUE $\mid$ FALSE $\mid$ [A-Z]+[0-9]+)    {[}A-Z0-9\_.{]}+                                                                             & 3 & A1MYNAME \\
NUMBER\_LITERAL & \begin{tabular}[c]{@{}l@{}}An integer, floating point\\     or scientific notation number literal\end{tabular} & [0-9]+ ,? [0-9]* (e [0-9]+)? & 0 & 10.1e12 \\
QUOTED\_FILE\_SHEET & A file reference within single quotes & '\textbackslash[ [0-9]+ \textbackslash] ([0-9A-Z\_ !@\#\$\%\textasciicircum{}\&*()\-+=\{\}:;$\mid$\textless\textgreater,./?\textbackslash{}\textbackslash{}] $\mid$ '')+ '!
& 5        & '{[}1{]}My Sheet'! \\
REFERENCE\_FUNCTION & Excel built-in reference function & (INDEX $\mid$ OFFSET $\mid$ INDIRECT)\textbackslash( & 5 & INDEX( \\
RESERVED\_NAME & An Excel reserved name & \_xlnm\textbackslash.  [A-Z\_]+ & -1 & \_xlnm.History \\
SHEET & The name of a worksheet &
	([0-9A-Z\_.]+ $\mid$ ' ([0-9A-Z\_ !@\#\$\%\textasciicircum{}\&*()\-+={}$\mid$:;\textless\textgreater,./?\textbackslash\textbackslash] $\mid$ '')+ ') !
& 5        & Sheet1!            \\
STRING & String literal & " ([\textasciicircum{} "] $\mid$ "")* " & 0        & "He Said: ""Hi"""  \\
UNOP\_PREFIX & Unary postfix operator & \% & 0 & \% \\
UNOP\_POSTFIX & Unary prefix operator & + $\mid$ - & 0 & -                  \\
UDF & User Defined Function & (\_xll\textbackslash.)? [A-Z0-9]+  ( & 4 & MyFunction( \\
VERTICAL\_RANGE & Range of columns & \$? [A-Z]+ : \$? [A-Z]+ & 0 & A:Z \\ 
\bottomrule
\end{tabular}
\end{table*}

\subsection{Production rules}

We define our production rules in Extended BNF syntax. The start symbol is $Start$. If one does not wish to accept array formula's and wishes to omit the `=' at the start of a formula $Formula$ can be used as a start symbol instead, or the $Start$ non-terminal can be adjusted.

\begin{grammar}
<Start> ::= '=' Formula | ArrayFormula


%<Error> ::= "ERROR"
%
%<String> ::= "STRING"
%
%<QuotedFileSheet> ::= "QUOTED_FILESHEET"
%
%<Sheet> ::= "SHEET"
%
%<MultipleSheets> ::= "MULTIPLE_SHEETS"

<Cell> ::= "CELL"
	  \alt CellReferenceFunction

%<File> ::= "FILE"

<NamedRange> ::= "NAMED_RANGE"
            \alt "NAMED_RANGE_COMBINED"

<Prefix> ::= "SHEET"
	\alt "FILE" "SHEET"
	\alt "FILE" `!'
	\alt "QUOTED_FILE_SHEET"
	\alt "MULTIPLE_SHEETS"
	\alt "FILE" "MULTIPLE_SHEETS"

<ReferenceItem> ::= Cell
	\alt "VERTICAL_RANGE"
	\alt "HORIZONTAL_RANGE"
	\alt NamedRange
	\alt "ERROR" 

<Reference> ::= ReferenceItem
	\alt Reference `:' Reference
	\alt Reference `\ ' Reference
	\alt `(' Reference `)'
	\alt Prefix ReferenceItem
              \alt Prefix + "UDF" + Arguments + `)'
              
<Formula> ::= Reference
         \alt "NUMBER"
         \alt "STRING"
         \alt FunctionCall
         \alt "BOOL"
         \alt "RESERVED_NAME"

<Argument> ::= Formula
\alt ArrayAsArgument
\alt $\epsilon$

<Arguments> ::= Argument
\alt Argument `,' Arguments

<ArrayAsArgument> ::= `(' Arguments `)'

<Function> ::= "FUNCTION"
	\alt "UDF"

<FunctionCall> ::= Function Arguments `)'
\alt "UNOP_PREFIX" Formula
\alt Formula "UNOP_POSTFIX"
\alt Formula "BINOP" Formula
\alt `(' Formula `)'

<CellReferenceFunction> ::= "REFERENCE_FUNCTION" Arguments `)'

<ArrayFormula> ::= `\{=' ArrayArguments `\}' - merge wih ArrayAsArgument?

<ArrayArguments> ::= ArrayArgument
 \alt ArrayArgument `,' ArrayArguments
 \alt ArrayArgument `;' ArrayArguments
 
<ArrayArgument> ::= Formula
 \alt $\epsilon$


\end{grammar}

\subsection{Precedence}

All operators in Excel are left-associative, including the exponentiation operator. In order for our grammar to be unambiguous a separate operator precedence needs to be defined, which can be found in Table \ref{table:operatorprec}

\begin{table}
\label{table:operatorprec}
\caption{Operator precedence in Excel Formulas}
\begin{tabular}{lr}
Operator                                                                & Precedence \\
 & higher = greater \\
= \textless \  \textgreater \  \textless= \  \textgreater= \  \textless\textgreater & 1          \\
\&                                                                      & 2          \\
+ - (binary)                                                            & 3          \\
* /                                                                     & 4          \\
\textasciicircum                                                        & 5          \\
\%                                                                      & 6          \\
+ - (unary)                                                             & 7          \\
: \texttt{\char32}                                                             & 8         
\end{tabular}
\end{table}

Putting the operator precedence directly into the grammar is relatively easy (and can be automated), but makes the grammar less clear and compact.

\subsection{Ambiguity}

Unfortunately our grammar is not fully unambiguous, because we made a trade-off between avoid ambiguity, simplicity of the grammar and usefulness of the resulting trees. The following ambiguities are present and must be manually resolved using the methods your parser generator provides:

\synt{FunctionCall} and \synt{ArrayAsArgument} can conflict when only one argument is present. \synt{Functioncall} is the correct parse. Example ambiguous sentence: \texttt{=F((1))}.

\synt{FunctionCall} and \synt{Reference} can conflict when a bracketed reference is present where any formula can be present. Either parse is correct as they are logically equivalent, but we recommend parsing this as \synt{FunctionCall}. Example ambiguous sentence: \texttt{=(A1)}.

\subsection{Internationalization}

\todo{Move to discussion maybe?}

Excel formulas differ depending on the language of the software. For example function arguments are separated by a semicolon instead of a comma in locales that use the comma as a decimal separator.

Our grammar is only for the default (English) locale. Grammars for other locales can be gotten by replacing delimiters, error values and function names by their localized versions.

It is worth noting that Excel will always save formulas in either a locale-independent binary format (Excel 2003 and earlier) or in its English version (Excel 2007 and later). When interacting with Excel through its API 2 versions of the formula can be read or written: the English version and the version in the current locale.

\subsection{Intersection operator}

\todo{Probably not worth its own section}

The intersection binary operator in Excel formulas is a single space, while the rest of the language is mostly whitespace independent.
Care must thus be taken on this point when implementing this grammar.

If your parser supports implicit operators the intersection operator can be implemented this way.
An example of implicit operators outside of Excel formulas is in calculus where multiplication is often omitted and $5a$ is equivalent to $5 \cdot a$ thus making $\cdot$ an implied operator.


\section{Evaluation}
The grammar is implemented in the Irony parser generator framework \footnote{https://irony.codeplex.com/} and the resulting parser been made available for download \footnote{link for download}. To extract formulas and feed them to the parser we built a tool that opens spreadsheets, reads all their cells and identifies which cells have the same formula in R1C1. Groups of cells that share the same formula in the R1C1 style are commonly referred to as belonging to the same sibling class. The tool then selects the first cell from each sibling class, and uses its formula string as input to the parser. It parses only one cell from each sibling class - the only differences between the formulas in a sibling class are the values of the references, so the structure of the produced abstract syntax trees is exactly the same. The abstract syntax trees are then traversed to generate a variety of information, for example, the references of cells to other cells, the functions that are used, the operations that are performed and the fixed numbers in the formulas.

To evaluate the grammar we use it to parse a total of ??? formulas. Those were found in the two major datasets available in the spreadsheet research community: The Euses dataset \cite{euses}, comprising of ?? spreadsheets and the 16.000 spreadsheets in the Enron email corpus \cite{enron}, which became available after the Enron company declared bankruptcy. In total, the spreadsheets form the two datasets include ??? formula cells. These are grouped into ??? sibling classes, and this is the number of formulas that were used as input to the parser.

A total of ??? formulas were parsed successfully, and ?? generate errors. (is this discussion here??)

how long it takes to process the datasets

\section{Wacky Grammar}
table with how often wacky syntax occurs

The grammar resulted from many cycles of parse errors, enrichments and refinements. There are parts of the grammar that cover a particularly complex set of structures, and these are the ones that are analyzed further in this section.

\subsection{References}
 
Spreadsheet formulas allow for many different types of references, including single cell references, cell ranges, horizontal or vertical ranges, named ranges and reference-returning build-in or user-defined functions. All of these references can be internal (in the same or in different sheets) or external. Syntactically, they can be expressed in a number of ways. The simplest case of a reference to a cell range can be expressed in any of the following ways:

\begin{eqnarray*}
&SUM&(A1:A2) \\
= &SUM&(Sheet1!A1:A2) \\
= &SUM&(Sheet1!A1:(A2)) \\
= &SUM&('Sheet1'!A1:A2) \\
= &SUM&(Sheet1!A1:Sheet1!A2) \\
= &SUM&(Sheet1!A1:'Sheet1'!A2) \\
\end{eqnarray*}

The <Reference> rule covers all those types of referencing expressions. It was the rule that was the most complex to devise to cover all edge cases in the two datasets.

\begin{grammar}
	<Reference> ::= \[[
	\begin{stack} '$($' <Reference> '$)$'\\ <Reference> \begin{stack} '$:$' \\ '$ $' \end{stack} <Reference> \\
	\begin{stack} \\ \begin{stack} \\ "FILE" \end{stack} \begin{stack} "SHEET" \\ "MULTIPLE_SHEETS" \end{stack} \\ "FILE" '$!$' \\ "QUOTED_FILE_SHEET" \end{stack}
	\begin{stack} \begin{stack} \begin{stack} "CELL" \\ "REFERENCE-FUNCTION" <Arguments> '$)$' \end{stack} \\ "VERTICAL-RANGE" \\ "HORIZONTAL-RANGE" \\ \begin{stack} "NAMED-RANGE" \\ "NAMED-RANGE-COMBINED" \end{stack} \\ "ERROR" \end{stack} \\ "UDF" <Arguments> '$)$'\end{stack}
	\end{stack}
	\]]
\end{grammar}
 
how often each case is used:
- SUM(Sheet2:Sheet3!A1)
- [1]!Hub\_Consolidation
- Functions returning references (INDEX|OFFSET|INDIRECT|???) SUM(A1:INDEX(A:A,1,1:1))
- External UDFs RIGHT([1]!SheetName(),(LEN([1]!SheetName())-20))
- Prefixed right limits =SUM(Deals!F9:'Deals'!F16)
- This thing we do not know what it is =BLP|M!'INDU Index,[PX\_close\_5d]'

\subsection{Arrays}
Array formulas
- IF(OR(MONTH(GQ2) = {3,9}), IF(ABS(SUM(GQ223) - GQ225)>0.001,1,0), 0)
Arrays used as arguments
- LARGE((F38,C38),1)

\subsection{Reserved names}
- Add-in functions C9*\_xll.HEAT(B9,C9)
- \_xlnm.* =SUMIF('150000'!A:A,\_xlnm.Print\_Titles,'150000'!C:C)


\subsection{Smelly grammar constructs}
 + how often they are used
- Implicit intersect operator =COUNT(A1:A10˽A1:A5)
- Complex ranges =SUM(I8:namedRange:M8)
- Combinations of the above SUM((Total\_Cost Jan):(Total\_Cost Apr.))

\section{Discussion and Limitations}
Grammar for Excel version ??? : old structures we do not support:
- Regular expressions Sheet1!A2 = SUM('S*'!A1) = SUM(Sheet2:Sheet3!A1) after Excel 2010
Not tested on array formulas
QuotedContent / ContentInFile were left out, discuss

\section{Comparison against the LibreOffice grammar}

\section{Related work}

\section{Conclusion}
The conclusion goes here.


\section*{Acknowledgment}
The authors would like to thank...


\bibliographystyle{IEEEtran}
\bibliography{XLGrammarRefs}



\end{document}


